\documentclass[12pt,a4paper]{article}

\usepackage[T1]{fontenc}
\usepackage[polish]{babel}
\usepackage[utf8]{inputenc}
\usepackage{lmodern}
\selectlanguage{polish}
\usepackage{graphicx}

\begin{document}
\pagenumbering{gobble}
\clearpage
\begin{figure}[h]
\centering
\includegraphics{ps-logo.png}
\end{figure}
\hspace{3cm}
\begin{center}Dokumentacja projektu\end{center}
\begin{center}2019/2020\end{center}
\hspace{3cm}
\begin{center}\large\textbf{Zarządzanie systemami informatycznymi}\end{center}
\begin{center}\large\textit{Moduł 1}\end{center}

\hspace{7cm}
\begin{flushright}Kierunek: Informatyka
\end{flushright}
\begin{flushright}Członkowie zespołu:
\par
\textit{Karol Nawrot}
\par
\textit{Piotr Sobieszczyk}
\end{flushright}
\vfill
\begin{center}Gliwice, 2019/2020\end{center}

\newpage
\pagenumbering{arabic}
\tableofcontents

\newpage
\section{Wprowadzenie}

\subsection{Role w projekcie}
1.Karol - Główny researcher zespołu, tworzenie prezentacji, project manager, pasjonat sieci komputerowych \newline
2.Piotr - Tworzenie repozytorium, dokumentacji, dobrze orientuje się w sieciach bezprzewodowych
\subsection{Cel projektu}
\newline
2.8.a.  Scharakteryzuj istniejący (dowolnie wybrany) system informatyczny wspomagający pracę podmiotu z sektora administracji publicznej [→https://www.rekord.com.pl/ratusz/]
\newline
2.8.b. Przedstaw (na przykładzie) praktyczne wykorzystanie narzędzia wspomagającego analizę systemu informatycznego poprzez badanie plików dziennika [→https://tiny.pl/t3f8c]
\newline
2.8.c. Przeanalizuj następujący materiał [→https://tiny.pl/t3ds8] pod kątem identyfikacji potencjału doskonalenia oraz zaproponuj zmiany doskonalące
\newpage

\section{Założenia projektowe}

\subsection{Założenia techniczne i nietechniczne}
Zakładamy, że jesteśmy połączeni z siecią bezprzewodową. \\
Posiadamy komputer lub laptop z systemem Windows.

\subsection{Stos technologiczny}
Narzędzia i systemy informatyczne związane z projektem \newline 
1. Przeglądarka internetowa dowolnej firmy
\newline

\subsection{Oczekiwane rezultaty projektu}
Oczekujemy, że pomyślnie zacharakteryzujemy system informatyczny https://www.rekord.com.pl/ratusz/. \newline
Znajdziemy również przykłady zastosowania narzędzia https://opensource.com/article/19/4/log-analysis-tools do badania plików dziennika. \newline Na koniec zaproponujemy również zmiany doskonalające do systemu informatycznego polis ubezpieczeniowych.
\newpage
\section{Realizacja projektu}
Opis wszystkich etapów realizacji projektu \\
1. Research \\
2. Sformułowanie założeń co do oczekiwanych rezultatów \\
3. Przeprowadzenie eksperymentów za pomocą podanych w stosie technologicznym narzędzi \\
4. Stwierdzenie wniosków oraz przedyskutowanie wyników \\
5. Prezentacja wyników \\
\newpage
\section{Wnioski}
\begin{itemize}
\item \textit{Spostrzeżenia} \\
Strona internetewa systemu wspomagającego pracę podmiotu z sektora administracji publicznej jest bardzo ładna. Grafik miał świetny gust. Po przeczytaniu całej zawartości stwierdzamy, iż narzędzie jest świetnym dodatkiem do codzienniej pracy jednostki samorządowej.Założenia dotyczące systemu polis również zostały dobrze sformułowane, ale widać tutaj potencjał na rozwój.Ogółem można rzec, iż nasza wiedza została poszerzona.
\item \textit{Osiągnięcia}\\
Poszerzyliśmy swoje horyzonty oraz ujrzeliśmy przykład przydatnego oprogramowania, w dodatku dobrze wykonanego, co jest niepodważalnie świetnym doświadczeniem. Zdobycie wiedzy.
\item \textit{Potencjał rozwoju}\\
Brakuje zabezpieczeń biometrycznych jako logowania do aplikacji. Analiza sposobu pisania jako dodatkowe zabezpieczenie również byłaby milewidziana.
\end{itemize}
\end{document}