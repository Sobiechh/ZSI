\documentclass[12pt,a4paper]{article}

\usepackage[T1]{fontenc}
\usepackage[polish]{babel}
\usepackage[utf8]{inputenc}
\usepackage{lmodern}
\selectlanguage{polish}
\usepackage{graphicx}

\begin{document}
\pagenumbering{gobble}
\clearpage
\begin{figure}[h]
\centering
\includegraphics{media/ps-logo.png}
\end{figure}
\hspace{3cm}
\begin{center}Dokumentacja projektu\end{center}
\begin{center}2019/2020\end{center}
\hspace{3cm}
\begin{center}\large\textbf{Zarządzanie systemami informatycznymi}\end{center}
\begin{center}\large\textit{Moduł 4}\end{center}

\hspace{7cm}
\begin{flushright}Kierunek: Informatyka
\end{flushright}
\begin{flushright}Członkowie zespołu:
\par
\textit{Karol Nawrot}
\par
\textit{Piotr Sobieszczyk}
\end{flushright}
\vfill
\begin{center}Gliwice, 2019/2020\end{center}

\newpage
\pagenumbering{arabic}
\tableofcontents

\newpage
\section{Wprowadzenie}

\subsection{Role w projekcie}
1.Karol - Główny researcher zespołu, tworzenie prezentacji, project manager, pasjonat sieci komputerowych \newline
2.Piotr - Tworzenie repozytorium, dokumentacji, dobrze orientuje się w sieciach bezprzewodowych


\subsection{Cel projektu}
Omówienie dobrych praktyk w zakresie wykorzystywania systemów informatycznych dla potrzeb pracy zdalnej.

\newpage

\section{Założenia projektowe}
Zakładamy, że jesteśmy połączeni z siecią bezprzewodową. \\
\subsection{Założenia techniczne i nietechniczne}
\ldots 

\subsection{Stos technologiczny}
1. Przeglądarka internetowa dowolnej firmy
2. Komputer

\subsection{Oczekiwane rezultaty projektu}
Oczekujemy, że dokładnie przedstawimy różne praktyki, które mogą pomagać programistom i nie tylko w pracy zdalnej. Sprawdzimy zalety i ewentualne wady wybranych rozwiązań.


\newpage
\section{Realizacja projektu}
Opis wszystkich etapów realizacji projektu \\
1. Research \\
2. Sformułowanie założeń co do oczekiwanych rezultatów \\
3. Przeprowadzenie eksperymentów za pomocą podanych w stosie technologicznym narzędzi \\
4. Stwierdzenie wniosków oraz przedyskutowanie wyników \\
5. Prezentacja wyników \\

\newpage
\section{Wnioski}

\begin{itemize}
\item \textit{Spostrzeżenia} \\
Po doczytaniu na ten temat stwierdzamy z Piotrkiem, ze jesteśmy na tak.
\item \textit{Osiągnięcia}
Wiedza została rozszerzona. Wiemy jak pracować zdalnie i jak robić to efektywnie, we dwójkę nam się dobrze teraz pracuje.
\item \textit{Potencjał rozwoju}
Ja bym dodał jeszcze szósty podpunkt do tej strony.
\end{itemize}
\end{document}