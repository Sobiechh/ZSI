\documentclass[12pt,a4paper]{article}

\usepackage[T1]{fontenc}
\usepackage[polish]{babel}
\usepackage[utf8]{inputenc}
\usepackage{lmodern}
\selectlanguage{polish}
\usepackage{graphicx}

\begin{document}
\pagenumbering{gobble}
\clearpage
\begin{figure}[h]
\centering
\includegraphics{ps-logo.png}
\end{figure}
\hspace{3cm}
\begin{center}Dokumentacja projektu\end{center}
\begin{center}2019/2020\end{center}
\hspace{3cm}
\begin{center}\large\textbf{Zarządzanie systemami informatycznymi}\end{center}
\begin{center}\large\textit{Moduł 1}\end{center}

\hspace{7cm}
\begin{flushright}Kierunek: Informatyka
\end{flushright}
\begin{flushright}Członkowie zespołu:
\par
\textit{Karol Nawrot}
\par
\textit{Piotr Sobieszczyk}
\end{flushright}
\vfill
\begin{center}Gliwice, 2019/2020\end{center}

\newpage
\pagenumbering{arabic}
\tableofcontents

\newpage
\section{Wprowadzenie}

\subsection{Role w projekcie}
1.Karol - Główny resaercher zespołu, project manager, pasjonat sieci komputerowych \newline
2.Piotr - Tworzenie repozytorium, prezentacji, dobrze orientuje się w sieciach bezprzewodowych
\subsection{Cel projektu}
\newline
1.5.a. Przykładowe wykorzystanie narzędzi informatycznych w analizie ruchu sieciowego [→Wireshark; →Ostinato]
\newline
1.5.b. Mechanizmy ochrony transmisji w sieciach bezprzewodowych [→https://www.wi-fi.org/discover-wi-fi/security]
\newline
1.5.c. Biometryczna kontrola dostępu [→https://www.typingdna.com/]

\newpage

\section{Założenia projektowe}

\subsection{Założenia techniczne i nietechniczne}
\ldots 

\subsection{Stos technologiczny}
Narzędzia i systemy informatyczne związane z projektem

Program Wireshark, służący do analizy ruchu sieciowego.

\subsection{Oczekiwane rezultaty projektu}
\ldots 
Oczekujemy, że będziemy mo
\newpage
\section{Realizacja projektu}
Opis wszystkich etapów realizacji projektu

\newpage
\section{Wnioski}

\begin{itemize}
\item \textit{Spostrzeżenia}
\item \textit{Osiągnięcia}
\item \textit{Potencjał rozwoju}
\end{itemize}
\end{document}